\chapter[Protocole essais brûleur ATR30 sur Calhory ]%
{Essais brûleur ATR30}
\section{Objectifs scientifiques}

 L'influence du swirl sur les flammes d'oxycombustion est étudiée à travers le brûleur des fours ATR.  L'objectif du stage est d'étudier la topologie de la flamme (longueur et forme principalement). Le procédé ATR assure la production d'un mélange $CO/H_{2}$  grâce à une oxycombustion riche et sous haute pression ($40 bar$). Dans le but d'étudier la combustion au CRCD, le travail réalisé a permis d'établir des lois d'échelles pour approcher les principales caractéristiques de la flamme tout en travaillant en combustion pauvre, à l'oxygène et à pression atmosphérique.
 
 Le travail réalisé consiste en deux axes majeurs :
 \begin{enumerate}
\item Démontrer 
\end{enumerate}