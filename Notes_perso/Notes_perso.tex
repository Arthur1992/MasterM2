L'objectif est que la R&D agisse seule avant d'avoir quelque chose d'intéressant à fournir à l'ingénierie.
L'ingénierie peut etre méfiante de la R&D, et a peur que ce soit juste du vol de technologies (si les chercheurs R&D vont chez le concurrent rapidement). Ainsi, il peuvent être prêt à écouter si on leur parle technologie et non recherche en amont, ça ne les intéresse pas.
Souvent, ils ont les plans mais ne sont pas à l'origine des études qui ont menées à ces plans.
Ils sont également extrêmement prudents et orientés contrats. Anecdote de Rémi : un brûleur syngaz était commandé pour 80MW, l'ingénierie ayant peur de faire des scale-up sur un modèle de 50MW, et ne pouvant pas garantir des résultats sur 80MW à ce prix ont fait un contrat avec 2*50MW, car ils étaient certains du résultat. Du coup, les concurrents étaient moins chers.
L'ingénierie ne prendra jamais aucun risque.

Auprès d'Air Liquide, le business premier est la fourniture de contrat O2. Il fut un temps ou la création de propane était une stratégie, sauf qu'on était en concurrence avec nos clients et le message n'était pas clair. Du coup AL a arrêté.

AL est prêt à mettre bcp d'argent si il y a des business derrière et si on arrive à convaincre 1 personne  de la direction avec un message clair. Autrement, ils laisseront "jouer" sur des petits projets sans mettre bcp d'argent. C'est le cas de la combustion pressurisée ou Cu-ss. Pour les convaincre, il faut adhérer au message molécule &business

Sur un plan personnel, énormément de bouche à oreil, positif comme négatif, donc ne jamais prendre démesurémennt ses aises et peut venir être cherché.

Les postes stratégiques sont tracés à l'avance INSEAD?

De mon côté :
ce que j'aime c'est l'industrie, être dans la convergence, pouvoir discuter le matin avec un technicien, puis discuter des axes stratégiques avec Remi, puis aller voir Bernard pour discuter d'un modèle de viscosité turbulente. 
J'aime les échéances et j'aime quand ça foire pour des raisons non prévues.
